\documentclass[11pt]{exam}
\usepackage{amsmath}
\usepackage{amssymb}
\usepackage{array}
\usepackage[bmargin=1.0in]{geometry}
\geometry{margin=0.75in}
\geometry{tmargin=0.75in}

\begin{document}
\thispagestyle{empty}

\begin{center}
\Large{MATH 1554 QH, Written Assignment 1}

\vspace{0.5cm}
\large{Niraj Khatri}

\vspace{0.5cm}
\small{September 2024}

\end{center}

\vspace{1cm}

\begin{questions}
  % Question 1
  \question[4] Consider the data in the table below.
  \begin{table}[h]\small
    \centering
      \begin{tabular}{|c|c|c|c|c|}
        \hline
        $x$ & -1 & 0 & 1 & 2 \\
        \hline
        $y(x)$ & -1 & 1.5 & 2 & -2.5 \\
        \hline
      \end{tabular}
  \end{table}
  \begin{parts}
    \part Construct an augmented matrix that can be used to compute the coefficients $a_0$, $a_1$, and $a_2$ of the polynomial $y(x) = a_0 + a_1 x + a_2x^2 + a_3x^3$ that passes through the points in the table above.
    \vspace{0.2cm}
    \begin{center}
        The augmented matrix for this system is:
    \end{center}
    \[
    \begin{bmatrix}
        1 & -1 & 1 & -1 & -1 \\
        1 & 0 & 0 & 0 & 1.5 \\
        1 & 1 & 1 & 1 & 2 \\
        1 & 2 & 4 & 8 & 2.5 \\
    \end{bmatrix}
    \]
    \part Reduce the augmented matrix that you constructed in part (a) to RREF.
    \vspace{0.2cm}
    \begin{center}
        The reduced row echelon form (RREF) of the matrix is:
    \end{center}
    \[
    \begin{bmatrix}
        1 & 0 & 0 & 0 & 1.5 \\
        0 & 1 & 0 & 0 & 2 \\
        0 & 0 & 1 & 0 & -1 \\
        0 & 0 & 0 & 1 & -0.5 \\
    \end{bmatrix}
    \]

    \part Clearly state the values of $a_0$, $a_1$, $a_2$, and $a_3$.
    
    \vspace{0.2cm}
    \begin{center}
        From the RREF matrix, we can see the values:
        \vspace{0.1cm}
        \begin{align*}
            a_0 &= {1.5} \\
            a_1 &= {2} \\
            a_2 &= {-1} \\
            a_3 &= {-0.5} \\
        \end{align*}
    \end{center}
  \end{parts}

\newpage % Ensures the next question starts on a new page

  % Question 2
  \question[5] Consider the linear system of equations below.
  \begin{align}
    x_1+x_2+2x_3 +7x_5 &= 20 \\
    x_2 + 3x_5 &= 1 \\
    x_3 +5x_5 &= 2
  \end{align}
  The variables in the system are $x_1$, $x_2$, $x_3$, $x_4$, and $x_5$.
  \begin{parts}
    \part Express the system as a $3\times6$ augmented matrix.
    \begin{solution}
      Your answer here.
    \end{solution}

    \part Reduce the system you constructed in part (a) to RREF.
    \begin{center}
        The reduced row reduced echelon form (RREF) of the matrix is:
    \end{center}

    \[
    \begin{bmatrix}
        1 & 0 & 0 & 0 & 1.5 \\
        0 & 1 & 0 & 0 & 2 \\
        0 & 0 & 1 & 0 & -1 \\
        0 & 0 & 0 & 1 & -.05 \\
    \end{bmatrix}
    \]

    \part Express the solution set of the linear system in parametric vector form.
    \begin{solution}
      Your answer here.
    \end{solution}
  \end{parts}

\newpage % Ensures the next question starts on a new page

  % Question 3
  \question[1] There are two parts to this question
  \begin{parts}
    \part Please state your name, facilitator, and high school (in case we need to get in contact with them for any reason). Your facilitator is someone at your high school.
    \begin{solution}
      Your answer here.
    \end{solution}

    \part Please confirm that you have followed all submission guidelines:
    \begin{enumerate}
      \item Your work is legible in the files you uploaded.
      \item Questions are answered in the order in which they were given.
      \item Each question is answered on its own page (or pages).
      \item Your work is submitted as a single PDF file.
      \item You uploaded your work to the correct location in Gradescope.
      \item During the upload process, you indicated which pages correspond to which question.
      \item None of your pages are upside down or sideways.
    \end{enumerate}
    \begin{solution}
      Your answer here.
    \end{solution}
  \end{parts}
\end{questions}

\vspace{1cm}

\end{document}
