\documentclass[11pt]{exam}
\usepackage{amsmath}
\usepackage{amssymb}
\usepackage{array}
\usepackage[bmargin=1.0in]{geometry}
\geometry{margin=0.75in}
\geometry{tmargin=0.75in}

\begin{document}
\thispagestyle{empty}

\begin{center}
\Large{MATH 1554 QH, Written Assignment 1}

\vspace{0.5cm}
\large{Niraj Khatri}

\vspace{0.5cm}
\small{September 2024}

\end{center}

\vspace{1cm}

\begin{questions}
  % Question 1
  \question[4] Consider the data in the table below.
  \begin{table}[h]\small
    \centering
      \begin{tabular}{|c|c|c|c|c|}
        \hline
        $x$ & -1 & 0 & 1 & 2 \\
        \hline
        $y(x)$ & -1 & 1.5 & 2 & -2.5 \\
        \hline
      \end{tabular}
  \end{table}
  \begin{parts}
    %\part Construct an augmented matrix that can be used to compute the coefficients $a_0$, $a_1$, and $a_2$ of the polynomial $y(x) = a_0 + a_1 x + a_2x^2 + a_3x^3$ that passes through the points in the table above.
    \vspace{0.2cm}
    \part
        From the table above, we can construct a system of linear equation; 4 equations to be exact.
        \begin{align*}
            y(-1) &= a_0+(-1)a_1+(-1)^2a_2+(-1)^3a_3=-1 \\
            y(0) &= a_0+(0)a_1+(0)^2a_2+(0)^3a_3=1.5 \\
            y(1) &= a_0+(1)a_1+(1)^2a_2+(1)^3a_3=2 \\
            y(2) &= a_0+(2)a_1+(2)^2a_2+(2)^3a_3=-2.5
        \end{align*}
        The augmented matrix for this system is:
    \[
    \begin{bmatrix}
        1 & -1 & 1 & -1 & -1 \\
        1 & 0 & 0 & 0 & 1.5 \\
        1 & 1 & 1 & 1 & 2 \\
        1 & 2 & 4 & 8 & -2.5 \\
    \end{bmatrix}
    \]
    \vspace{0.2cm}
    %\part Reduce the augmented matrix that you constructed in part (a) to RREF.
    \vspace{0.2cm}
    \part We can apply a series of row manipulations to effectively row reduce our matrix. Here are the key steps:

\begin{align*}
\begin{bmatrix}
    1 & -1 & 1 & -1 & -1 \\
    1 & 0 & 0 & 0 & 1.5 \\
    1 & 1 & 1 & 1 & 2 \\
    1 & 2 & 4 & 8 & -2.5
\end{bmatrix}
&\xrightarrow{R_2 - R_1, R_3 - R_1, R_4 - R_1}
\begin{bmatrix}
    1 & -1 & 1 & -1 & -1 \\
    0 & 1 & -1 & 1 & 2.5 \\
    0 & 2 & 0 & 2 & 3 \\
    0 & 3 & 3 & 9 & -1.5
\end{bmatrix} \\[12pt]
&\xrightarrow{R_3 - 2R_2, R_4 - 3R_2}
\begin{bmatrix}
    1 & -1 & 1 & -1 & -1 \\
    0 & 1 & -1 & 1 & 2.5 \\
    0 & 0 & 2 & 0 & -2 \\
    0 & 0 & 6 & 6 & -9
\end{bmatrix} \\[12pt]
&\xrightarrow{\frac{1}{2}R_3, R_4 - 3R_3}
\begin{bmatrix}
    1 & -1 & 1 & -1 & -1 \\
    0 & 1 & -1 & 1 & 2.5 \\
    0 & 0 & 1 & 0 & -1 \\
    0 & 0 & 0 & 6 & -3
\end{bmatrix} \\[12pt]
\end{align*}
\begin{align*}
&\xrightarrow{\frac{1}{6}R_4}
\begin{bmatrix}
    1 & -1 & 1 & -1 & -1 \\
    0 & 1 & -1 & 1 & 2.5 \\
    0 & 0 & 1 & 0 & -1 \\
    0 & 0 & 0 & 1 & -0.5
\end{bmatrix} \\[12pt]
&\xrightarrow{R_1 + R_4, R_2 - R_4}
\begin{bmatrix}
    1 & -1 & 1 & 0 & -1.5 \\
    0 & 1 & -1 & 0 & 3 \\
    0 & 0 & 1 & 0 & -1 \\
    0 & 0 & 0 & 1 & -0.5
\end{bmatrix} \\[12pt]
&\xrightarrow{R_1 + R_3, R_2 + R_3}
\begin{bmatrix}
    1 & -1 & 0 & 0 & -0.5 \\
    0 & 1 & 0 & 0 & 2 \\
    0 & 0 & 1 & 0 & -1 \\
    0 & 0 & 0 & 1 & -0.5
\end{bmatrix} \\[12pt]
&\xrightarrow{R_1 + R_2}
\begin{bmatrix}
    1 & 0 & 0 & 0 & 1.5 \\
    0 & 1 & 0 & 0 & 2 \\
    0 & 0 & 1 & 0 & -1 \\
    0 & 0 & 0 & 1 & -0.5
\end{bmatrix}
\vspace{0.2cm}
\end{align*}
    \part From the RREF matrix, we can see the values:
    \begin{center}
        \begin{align*}
            a_0 &= {1.5} \\
            a_1 &= {2} \\
            a_2 &= {-1} \\
            a_3 &= {-0.5} 
        \end{align*}
        Therefore, the polynomial that passes through the given points is:
        \begin{align*}
            y(x)&=1.5+2x-x^2-0.5x^3
        \end{align*}
    \end{center}
  \end{parts}

\newpage % Ensures the next question starts on a new page

  % Question 2
\question[5] Consider the linear system of equations below.
\begin{align}
  x_1 + x_2 + 2x_3 + 7x_5 &= 20 \\
  x_2 + 3x_5 &= 1 \\
  x_3 + 5x_5 &= 2
\end{align}
The variables in the system are $x_1$, $x_2$, $x_3$, $x_4$, and $x_5$.
\begin{parts}
  \vspace{0.2cm}
  \part The $3\times6$ augmented matrix of the system is:
  \[
  \begin{bmatrix}
      1 & 1 & 2 & 0 & 7 & 20 \\
      0 & 1 & 0 & 0 & 3 & 1 \\
      0 & 0 & 1 & 0 & 5 & 2 
  \end{bmatrix}
  \]
  \vspace{0.2cm}
  \part Let's perform the row reduction to bring the matrix into RREF. Here are the steps:

\begin{align*}
\begin{bmatrix}
    1 & 1 & 2 & 0 & 7 & 20 \\
    0 & 1 & 0 & 0 & 3 & 1 \\
    0 & 0 & 1 & 0 & 5 & 2 
\end{bmatrix}
&\xrightarrow{R_1 - 2R_3}
\begin{bmatrix}
    1 & 1 & 0 & 0 & -3 & 16 \\
    0 & 1 & 0 & 0 & 3 & 1 \\
    0 & 0 & 1 & 0 & 5 & 2
\end{bmatrix} \\[12pt]
&\xrightarrow{R_1 - R_2}
\begin{bmatrix}
    1 & 0 & 0 & 0 & -6 & 15 \\
    0 & 1 & 0 & 0 & 3 & 1 \\
    0 & 0 & 1 & 0 & 5 & 2
\end{bmatrix}
\end{align*}

Thus, the matrix in RREF is:
\[
\begin{bmatrix}
    1 & 0 & 0 & 0 & -6 & 15 \\
    0 & 1 & 0 & 0 & 3 & 1 \\
    0 & 0 & 1 & 0 & 5 & 2 
\end{bmatrix}
\]
\vspace{0.2cm}

\part Express the solution set of the linear system in parametric vector form.
\vspace{0.2cm}
\begin{center}
  From the RREF matrix, we can see that $x_1$, $x_2$, and $x_3$ are basic variables, while $x_4$ and $x_5$ are free variables. The solution set can be written as:
  \[
  \begin{pmatrix}
  x_1 \\
  x_2 \\
  x_3 \\
  x_4 \\
  x_5
  \end{pmatrix}
  =
  \begin{pmatrix}
  15 \\
  1 \\
  2 \\
  0 \\
  0
  \end{pmatrix}
  + x_5 
  \begin{pmatrix}
  -6 \\
  3 \\
  5 \\
  0 \\
  1
  \end{pmatrix}
  \]
\end{center}
\end{parts}


\newpage % Ensures the next question starts on a new page

  % Question 3
  \question[1] There are two parts to this question
\begin{parts}
\part \mbox{} %\vspace{0.2cm}
\begin{center}
\begin{tabular}{rl}
    Name: & Niraj Khatri \\
    Facilitator: & Jamie Hamrick (jamie.hamrick@cobbk12.org) \\
    School: & Pope High School
\end{tabular}
\end{center}
\vspace{0.2cm}

    \part Please confirm that you have followed all submission guidelines:
    \begin{enumerate}
      \item Your work is legible in the files you uploaded.
      \item Questions are answered in the order in which they were given.
      \item Each question is answered on its own page (or pages).
      \item Your work is submitted as a single PDF file.
      \item You uploaded your work to the correct location in Gradescope.
      \item During the upload process, you indicated which pages correspond to which question.
      \item None of your pages are upside down or sideways.
    \end{enumerate}
    \begin{center}
        YES
    \end{center}
  \end{parts}
\end{questions}

\vspace{1cm}

\end{document}
