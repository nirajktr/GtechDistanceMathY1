\documentclass[11pt]{exam}
\usepackage{amsmath}
\usepackage{amssymb}
\usepackage{array}
\usepackage[bmargin=1in, tmargin=0.75in, lmargin=0.75in, rmargin=0.75in]{geometry}

\begin{document}
\thispagestyle{empty}
\begin{center}
    \Large{MATH 1554 QH, Written Assignment 4}
    \vspace{0.5cm}
    
    \large{Niraj Khatri}
    \vspace{0.5cm}
    
    \small{November 2024}
\end{center}
\vspace{1cm}

\begin{questions}
\question[5]
Consider the set of vectors in $\mathbb{R}^3$:
\[
W = \left\{ \begin{pmatrix} 1 \\ 2 \\ 2 \end{pmatrix}, \begin{pmatrix} 3 \\ 0 \\ 3 \end{pmatrix}, \begin{pmatrix} 7 \\ -1 \\ 2 \end{pmatrix} \right\}
\]

\begin{parts}
\part Let's apply the Gram-Schmidt process to convert $W$ into an orthogonal set $U$:

First vector: 
\[
\mathbf{u}_1 = \mathbf{w}_1 = \begin{pmatrix} 1 \\ 2 \\ 2 \end{pmatrix}
\]

Second vector:
\begin{align*}
\mathbf{u}_2 &= \mathbf{w}_2 - \text{proj}_{\mathbf{u}_1}(\mathbf{w}_2) \\
\text{proj}_{\mathbf{u}_1}(\mathbf{w}_2) &= \frac{\mathbf{w}_2 \cdot \mathbf{u}_1}{\mathbf{u}_1 \cdot \mathbf{u}_1}\mathbf{u}_1 \\
&= \frac{9}{9}\begin{pmatrix} 1 \\ 2 \\ 2 \end{pmatrix} = \begin{pmatrix} 1 \\ 2 \\ 2 \end{pmatrix} \\
\mathbf{u}_2 &= \begin{pmatrix} 3 \\ 0 \\ 3 \end{pmatrix} - \begin{pmatrix} 1 \\ 2 \\ 2 \end{pmatrix} = \begin{pmatrix} 2 \\ -2 \\ 1 \end{pmatrix}
\end{align*}

Third vector:
\begin{align*}
\mathbf{u}_3 &= \mathbf{w}_3 - \text{proj}_{\mathbf{u}_1}(\mathbf{w}_3) - \text{proj}_{\mathbf{u}_2}(\mathbf{w}_3) \\
\text{proj}_{\mathbf{u}_1}(\mathbf{w}_3) &= \frac{\mathbf{w}_3 \cdot \mathbf{u}_1}{\mathbf{u}_1 \cdot \mathbf{u}_1}\mathbf{u}_1 = \begin{pmatrix} 1 \\ 2 \\ 2 \end{pmatrix} \\
\text{proj}_{\mathbf{u}_2}(\mathbf{w}_3) &= \frac{\mathbf{w}_3 \cdot \mathbf{u}_2}{\mathbf{u}_2 \cdot \mathbf{u}_2}\mathbf{u}_2 = \begin{pmatrix} 4 \\ -4 \\ 2 \end{pmatrix} \\
\mathbf{u}_3 &= \begin{pmatrix} 7 \\ -1 \\ 2 \end{pmatrix} - \begin{pmatrix} 1 \\ 2 \\ 2 \end{pmatrix} - \begin{pmatrix} 4 \\ -4 \\ 2 \end{pmatrix} = \begin{pmatrix} 2 \\ 1 \\ -2 \end{pmatrix}
\end{align*}

\part Let's normalize the vectors in $U$:

For $\mathbf{u}_1$:
\[
\|\mathbf{u}_1\| = \sqrt{1^2 + 2^2 + 2^2} = 3 \quad \Rightarrow \quad \hat{\mathbf{e}}_1 = \begin{pmatrix} 1/3 \\ 2/3 \\ 2/3 \end{pmatrix}
\]

For $\mathbf{u}_2$:
\[
\|\mathbf{u}_2\| = \sqrt{2^2 + (-2)^2 + 1^2} = 3 \quad \Rightarrow \quad \hat{\mathbf{e}}_2 = \begin{pmatrix} 2/3 \\ -2/3 \\ 1/3 \end{pmatrix}
\]

For $\mathbf{u}_3$:
\[
\|\mathbf{u}_3\| = \sqrt{2^2 + 1^2 + (-2)^2} = 3 \quad \Rightarrow \quad \hat{\mathbf{e}}_3 = \begin{pmatrix} 2/3 \\ 1/3 \\ -2/3 \end{pmatrix}
\]

\part The $Q$ matrix is:
\[
Q = \begin{bmatrix} 
1/3 & 2/3 & 2/3 \\
2/3 & -2/3 & 1/3 \\
2/3 & 1/3 & -2/3
\end{bmatrix}
\]

\part The $R$ matrix is given by $Q^TA$: \\
where $A$ =
\[
\begin{bmatrix}
    1 & 3 & 7 \\
    2 & 0 & -1 \\
    2 & 3 & 2
\end{bmatrix}
\]
\[
R = \begin{bmatrix} 
1/3 & 2/3 & 2/3 \\
2/3 & -2/3 & 1/3 \\
2/3 & 1/3 & -2/3
\end{bmatrix}
\begin{bmatrix}
    1 & 3 & 7 \\
    2 & 0 & -1 \\
    2 & 3 & 2
\end{bmatrix}=
\begin{bmatrix}
    3 & 3 & 3 \\
    0 & 3 & 6 \\
    0 & 0 & 3
\end{bmatrix}
\
\]
\end{parts}
\newpage
\question[5]
Consider the points $P(-1,0,1)$, $Q(0,-1,0)$, $R(1,0,1)$, and $S(0,1,4)$.

\begin{parts}
\part The over-determined system $A\vec{x} = \vec{z}$ is:
\[
\begin{bmatrix}
-1 & 0 & 1 \\
0 & -1 & 1 \\
1 & 0 & 1 \\
0 & 1 & 1
\end{bmatrix}
\begin{pmatrix}
a \\ b \\ c
\end{pmatrix} =
\begin{pmatrix}
1 \\ 0 \\ 1 \\ 4
\end{pmatrix}
\]

\part The normal equations $(A^TA)\vec{x} = A^T\vec{z}$ give us:
\[
\begin{bmatrix}
2 & 0 & 0 \\
0 & 2 & 0 \\
0 & 0 & 4
\end{bmatrix}
\begin{pmatrix}
a \\ b \\ c
\end{pmatrix} =
\begin{pmatrix}
0 \\ 3 \\ 6
\end{pmatrix}
\]
\[
A^TA = \begin{bmatrix}
    -1 & 0 & 1 & 0 \\
    0 & -1 & 0 & 1 \\
    1 & 1 & 1 & 1
\end{bmatrix}
\begin{bmatrix}
    -1 & 0 & 1 \\
    0 & -1 & 1 \\
    1 & 0 & 1 \\
    0 & 1 & 1 
\end{bmatrix}
= 
\begin{bmatrix}
    2 & 0 & 0 \\
    0 & 2 & 0 \\
    0 & 0 & 4
\end{bmatrix}
\]
\[
A^Tz = \begin{bmatrix}
    -1 & 0 & 1 & 0 \\
    0 & -1 & 0 & 1 \\
    1 & 1 & 1 & 1
\end{bmatrix}
\begin{bmatrix}
    1 \\
    0 \\
    1 \\
    4
\end{bmatrix}
= \begin{bmatrix}
    0 \\
    4 \\
    6
\end{bmatrix}
\]


Solving the system:
\[
\begin{bmatrix}
    2 & 0 & 0 \\
    0 & 2 & 0 \\
    0 & 0 & 4
\end{bmatrix}
\begin{bmatrix}
    x_1 \\
    x_2 \\
    x_3
\end{bmatrix}
= \begin{bmatrix}
    0 \\
    4 \\
    6
\end{bmatrix}
\]
gives us:
\begin{align*}
x_1 &= 0 \\
x_2 &= 2 \\
x_3 &= \frac{3}{2}
\end{align*}

\part The equation of the best-fit plane is:
\[
z =  2y + \frac{3}{2}
\]
or equivalently:
\[
z = 0x + 2y + 1.5
\]
\end{parts}
\end{questions}
\end{document}