\documentclass[11pt]{exam}
\usepackage{amsmath} % allows for align environment
\usepackage{amssymb} %
\usepackage{array} % for table alignments
% FONT FORMAT
% \renewcommand*\rmdefault{ppl} % change font to Palatino
\renewcommand*\rmdefault{lmss} % change font to lat mod ss
% ADJUST MARGINS
\usepackage[bmargin=1.0in]{geometry}
\geometry{margin=.75in}
\geometry{tmargin=.75in}
\begin{document}
\begin{center}
\Large MATH 1554 QH \\[2pt] Written Assignment 4
\end{center}
\thispagestyle{empty} % suppress page numbering
\noindent Please \textbf{show your work} for each of the questions below. It is
ok to use a calculator to check that your calculations that you made by hand were
correct, but it should be clear how you obtained each step in your work without use
of a calculator.
\begin{questions}
\question[5]
Consider the following set of vectors in $\mathbb{R}^3$:
\[
W = \left\{ \begin{pmatrix} 1 \\ 2 \\ 2 \end{pmatrix}, \begin{pmatrix} 3 \\
0 \\ 3 \end{pmatrix}, \begin{pmatrix} 7 \\ -1 \\ 2 \end{pmatrix} \right\}.
\]
\begin{parts}
\part Apply the Gram-Schmidt process to convert the set of vectors in \
( W \) into an orthogonal set of vectors \( U \). Show all steps clearly and
explicitly compute each orthogonal vector.
\part Normalize the vectors in \( U \) to produce an orthonormal set of
vectors.
\part Using the orthonormal set of vectors from part (b), construct the \
( Q \) matrix for the QR factorization of \( A \), where \( A \) is the matrix
whose columns are the vectors in \( W \).
\part Using the \( Q \) matrix from part (c), construct the corresponding
upper triangular \( R \) matrix in the QR factorization.
\end{parts}
\question[5]
Consider the following four points: \( P(-1, 0, 1) \), \(Q (0, -1, 0) \), \
( R(1, 0, 1) \), and \( S(0, 1, 4) \). We want to fit these points to the plane:
\[
z = ax + by + c.
\]
\begin{parts}
\part Set up an over-determined system \( A\vec{x} = \vec{z} \), where \( \
vec{x} = \begin{pmatrix} a \\ b \\ c \end{pmatrix} \) contains the unknown
coefficients and \( A \) is the data matrix constructed from the given points.
\part Solve the normal equations to find the least-squares solution for \
( a \), \( b \), and \( c \).
\part Use your results from the previous part to determine the plane that
best fits the given points. In other words, write down the equation of the plane
that best fits the given data.
\end{parts}
\end{questions}
\noindent Please ensure that you follow the instructions below.
\begin{itemize}
\item Your work is legible in the files you uploaded.
\item Questions are answered in the order in which they were given.
\item During the upload process, you indicated which pages correspond to which
question.
\item During the upload process that none of your pages are upside down or
sideways.
\item Each question is answered on its own page (or pages).
\item Your work is submitted as a single PDF file.
\item You uploaded your work to the correct location in Gradescope (in other
words, make sure that you upload your work for this assignment to the correct
assignment).
\end{itemize}
\noindent Note that
\begin{itemize}
\item A small amount of points can be deducted for not following the
instructions above.
\item You can also change the orientation of the pages when you upload in
Gradescope.
\item Ensuring that these criteria are met helps ensure that your work is
graded efficiently and accurately.
\end{itemize}
\end{document}