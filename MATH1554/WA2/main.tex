\documentclass[11pt]{exam}
\usepackage{amsmath}
\usepackage{amssymb}
\usepackage{array}
\usepackage[bmargin=1in, tmargin=0.75in, lmargin=0.75in, rmargin=0.75in]{geometry}

\begin{document}
\thispagestyle{empty}

\begin{center}
    \Large{MATH 1554 QH, Written Assignment 2}

    \vspace{0.5cm}
    \large{Niraj Khatri}

    \vspace{0.5cm}
    \small{September 2024}
\end{center}

\vspace{1cm}

\begin{questions}

  % Question 1
  \question[5] Consider the matrix $A$ and vector $b$ below.
    \[
    A = \begin{bmatrix} 
    4 & 3 & 0 \\ 
    3 & 2 & 1 \\ 
    0 & 1 & 2 
    \end{bmatrix}, 
    \quad b = \begin{bmatrix} 
    7 \\ 
    8 \\ 
    3 
    \end{bmatrix}
    \]
    The system $Ax = b$, where $x \in \mathbb{R}^3$, can be solved using the LU factorization.
    \vspace{0.2cm}
  \begin{parts}
    \part
    We first factor $A$ into $LU$, where $L$ is a lower triangular matrix, and $U$ is an echelon form of matrix $A$. To obtain $U$, an echelon form of $A$, we can apply a series of row replacement operations:
    \begin{align*}
    A = \begin{bmatrix}
        4 & 3 & 0 \\
        3 & 2 & 1 \\
        0 & 1 & 2 \\
    \end{bmatrix}
    &\xrightarrow{R_2 = R_2 - (3/4)R_1}
    \begin{bmatrix}
        4 & 3 & 0 \\
        0 & \frac{-1}{4} & 1 \\
        0 & 1 & 2 \\
    \end{bmatrix} \\
    &\xrightarrow{R_3=R_3+(4)R_2}
    \begin{bmatrix}
        4 & 3 & 0 \\
        0 & \frac{-1}{4} & 1 \\
        0 & 0 & 6 \\
    \end{bmatrix}
\end{align*}

    From the row operations done above, the corresponding $L$$U$ matrix product is:
    \[
    A = LU = \begin{bmatrix}
    1 & 0 & 0 \\
    \frac{3}{4} & 1 & 0 \\
    0 & -4 & 1
    \end{bmatrix}
    \begin{bmatrix}
        4 & 3 & 0 \\
        0 & \frac{-1}{4} & 1 \\
        0 & 0 & 6 \\
    \end{bmatrix}
    \]
    \part
    Now solve $Ly = b$. Substituting $L$ and $b$, we have:
    \[
    \begin{bmatrix}
    1 & 0 & 0 \\
    \frac{3}{4} & 1 & 0 \\
    0 & -4 & 1
    \end{bmatrix}
    \begin{bmatrix}
    y_1 \\
    y_2 \\
    y_3
    \end{bmatrix}
    = \begin{bmatrix}
    7 \\
    8 \\
    3
    \end{bmatrix}
    \]
    Solving this, we get:
    \[
    y_1 = 7, \quad y_2 = \frac{11}{4}, \quad y_3 = 14
    \]

    \part

    Now, solve for $x$ using $Ux = y$. Substituting $U$ and the solved values of $y$:
    \[
    \begin{bmatrix}
    4 & 3 & 0 \\
    0 & -\frac{1}{4} & 1 \\
    0 & 0 & 6
    \end{bmatrix}
    \begin{bmatrix}
    x_1 \\
    x_2 \\
    x_3
    \end{bmatrix}
    = \begin{bmatrix}
    7 \\
    \frac{11}{4} \\
    14 \\
    \end{bmatrix}
    \]
    Solving this, we get:
    \[
    x_1 = 3, \quad x_2 = -\frac{5}{3}, \quad x_3 = \frac{7}{3} 
    \]
  \end{parts}

  \newpage

  % Question 2
  \question[5] Triangle $S$ is determined by the points $P(1,1), Q(3,1), R(1,2)$. Transform $T$ rotates points counterclockwise about the point $(0,1)$ by $\pi/2$ radians. 
  \vspace{0.2cm}
  \begin{parts}
    \part

    The points $P(1,1)$, $Q(3,1)$, and $R(1,2)$ can be represented in homogeneous coordinates as:
    \[
    D = \begin{bmatrix}
    1 & 3 & 1 \\
    1 & 1 & 2 \\
    1 & 1 & 1
    \end{bmatrix}
    \]

    \part 

    The transformation matrix for rotating points counterclockwise by $\pi/2$ radians about the point $(0,1)$ is:
    \[
    A = \begin{bmatrix}
    1 & 0 & 0 \\
    0 & 1 & 1 \\
    0 & 0 & 1
    \end{bmatrix}
    \begin{bmatrix}
    0 & -1 & 1 \\
    1 & 0 & 0 \\
    0 & 0 & 1
    \end{bmatrix}
    = 
    \begin{bmatrix}
    0 & -1 & 1 \\
    1 & 0 & 1 \\
    0 & 0 & 1
    \end{bmatrix}
    \]

    \part 

    The new coordinates of the transformed points can be calculated as $A \cdot D$:
    \[
    A \cdot D = \begin{bmatrix}
    0 & -1 & 1 \\
    1 & 0 & 1 \\
    0 & 0 & 1
    \end{bmatrix}
    \begin{bmatrix}
    1 & 3 & 1 \\
    1 & 1 & 2 \\
    1 & 1 & 1
    \end{bmatrix}
    = \begin{bmatrix}
    0 & 0 & -1 \\
    2 & 4 & 2 \\
    1 & 1 & 1
    \end{bmatrix}
    \]

    \part

    The coordinates of the triangle after the transformation are: \\
    $P^\prime = (0,2)$ \\
    $Q^\prime = (0, 4)$ \\
    $R^\prime = (-1,2)$ 
  \end{parts}

  \newpage

\end{questions}

\end{document}
