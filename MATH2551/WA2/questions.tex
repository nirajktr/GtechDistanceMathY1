\documentclass[12pt]{exam}

\usepackage{amsmath} % allows for align environment
\usepackage{amssymb} % 
\usepackage{array} % for table alignments
\usepackage{enumitem} % for enumerated lists
\usepackage{graphicx} % for inserting images



\begin{document}


\begin{center}
    \Large{MATH 2551 QH, Written Assignment 2}
\end{center}

\begin{questions}

    \question[5] Consider the function
\[
f(x,y)= x^2+y^2-3x-xy
\]

\begin{enumerate}[label=(\alph*)]
    \item  Identify the locations of all critical points of \( f \).
    
    \item  Use the second derivative test to classify each critical point as a local maximum, local minimum, or saddle point.

    \item Determine the absolute maximum and minimum values of $f$ on the curve $x^2+y^2=9$. 
    
    \item  Use your results from the previous parts of this question to determine the absolute maximum and minimum values of \( f \) on the closed disk 
    \[
    D=\{(x,y) \mid x^2+y^2\le9\}.
    \]
\end{enumerate}

\bigskip

    \question[5] Consider the parabola
\[
y^2 = 4x,
\]
and let \(P = (2,1)\) be a fixed point. The goal is to find the point on the parabola, $Q$, that is closest to \(P\) using the method of Lagrange multipliers, and then verify that the shortest distance is achieved along a line normal to the parabola.

\bigskip

\begin{enumerate}[label=(\alph*)]
    \item  
    Define the squared distance function from a point \((x,y)\) on the parabola to \(P\).

    \item  Use the method of Lagrange multipliers to determine the candidate point(s) on the parabola that minimizes the distance between \(P\) and the parabola. 

    \item
    Show that the line joining \(P\) to the point(s), $Q$, on the parabola found in part (b) is normal to the parabola. 

    \item Provide a suitable sketch for the problem. Please include the parabola \(y^2 = 4x\), the point $P$, the point $Q$, the line connecting $P$ and $Q$, and the tangent line to the parabola at $P$. Your graph should have axes labeled and it should be clear that the tangent and normal lines are perpendicular to each other. 

\end{enumerate}

\end{questions}




\end{document}
