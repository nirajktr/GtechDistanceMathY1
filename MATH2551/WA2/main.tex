\documentclass[11pt]{exam}
\usepackage{graphicx}
\usepackage{amsmath}
\usepackage{amssymb}
\usepackage{array}
\usepackage[bmargin=1in, tmargin=0.75in, lmargin=0.75in, rmargin=0.75in]{geometry}

\begin{document}
\thispagestyle{empty}
\begin{center}
    \Large{MATH 2551, Written Assignment 2}
    \vspace{0.5cm}
    
    \large{Niraj Khatri}
    \vspace{0.5cm}
    
    \small{February 2025}
\end{center}
\vspace{1cm}

\begin{questions}
\question Consider the function
\[
f(x,y)= x^2+y^2-3x-xy.
\]

\begin{parts}
\part Identify the locations of all critical points of \( f \).

To find the critical points, compute the partial derivatives of \( f \) with respect to \( x \) and \( y \), and set them equal to zero:
\[
f_x = \frac{\partial f}{\partial x} = 2x - 3 - y = 0,
\]
\[
f_y = \frac{\partial f}{\partial y} = 2y - x = 0.
\]
Solve the system of equations:
1. From \( f_y = 0 \), we get \( x = 2y \).
2. Substitute \( x = 2y \) into \( f_x = 0 \):
   \[
   2(2y) - 3 - y = 0 \implies 4y - 3 - y = 0 \implies 3y - 3 = 0 \implies y = 1.
   \]
3. Substitute \( y = 1 \) into \( x = 2y \):
   \[
   x = 2(1) = 2.
   \]
Thus, the only critical point is \( (2, 1) \).

\part Use the second derivative test to classify the critical point.

Compute the second partial derivatives:
\[
f_{xx} = \frac{\partial^2 f}{\partial x^2} = 2,
\]
\[
f_{yy} = \frac{\partial^2 f}{\partial y^2} = 2,
\]
\[
f_{xy} = \frac{\partial^2 f}{\partial x \partial y} = -1.
\]
The discriminant \( D \) is given by:
\[
D = f_{xx} f_{yy} - (f_{xy})^2 = (2)(2) - (-1)^2 = 4 - 1 = 3.
\]
Since \( D > 0 \) and \( f_{xx} > 0 \), the critical point \( (2, 1) \) is a **local minimum**.

\part Determine the absolute maximum and minimum values of \( f \) on the curve \( x^2 + y^2 = 9 \).

Use the method of Lagrange multipliers. Define the constraint:
\[
g(x, y) = x^2 + y^2 - 9 = 0.
\]
The gradients of \( f \) and \( g \) must satisfy:
\[
\nabla f = \lambda \nabla g.
\]
Compute the gradients:
\[
\nabla f = (2x - 3 - y, 2y - x),
\]
\[
\nabla g = (2x, 2y).
\]
Set up the equations:
\[
2x - 3 - y = \lambda (2x),
\]
\[
2y - x = \lambda (2y).
\]
Solve for \( \lambda \):
From the first equation:
\[
\lambda = \frac{2x - 3 - y}{2x}.
\]
From the second equation:
\[
\lambda = \frac{2y - x}{2y}.
\]
Set the two expressions for \( \lambda \) equal:
\[
\frac{2x - 3 - y}{2x} = \frac{2y - x}{2y}.
\]
Cross-multiply and simplify:
\[
(2x - 3 - y)(2y) = (2y - x)(2x),
\]
\[
4xy - 6y - 2y^2 = 4xy - 2x^2,
\]
\[
-6y - 2y^2 = -2x^2,
\]
\[
2x^2 - 2y^2 - 6y = 0,
\]
\[
x^2 - y^2 - 3y = 0.
\]
Use the constraint \( x^2 + y^2 = 9 \) to substitute \( x^2 = 9 - y^2 \):
\[
(9 - y^2) - y^2 - 3y = 0,
\]
\[
9 - 2y^2 - 3y = 0,
\]
\[
2y^2 + 3y - 9 = 0.
\]
Solve the quadratic equation:
\[
y = \frac{-3 \pm \sqrt{9 + 72}}{4} = \frac{-3 \pm \sqrt{81}}{4} = \frac{-3 \pm 9}{4}.
\]
Thus:
\[
y = \frac{6}{4} = \frac{3}{2} \quad \text{or} \quad y = \frac{-12}{4} = -3.
\]
For \( y = \frac{3}{2} \):
\[
x^2 = 9 - \left(\frac{3}{2}\right)^2 = 9 - \frac{9}{4} = \frac{27}{4} \implies x = \pm \frac{3\sqrt{3}}{2}.
\]
For \( y = -3 \):
\[
x^2 = 9 - (-3)^2 = 0 \implies x = 0.
\]

Evaluate \( f \) at these points:
1. \( \left(\frac{3\sqrt{3}}{2}, \frac{3}{2}\right) \):
   \[
   f = \left(\frac{3\sqrt{3}}{2}\right)^2 + \left(\frac{3}{2}\right)^2 - 3\left(\frac{3\sqrt{3}}{2}\right) - \left(\frac{3\sqrt{3}}{2}\right)\left(\frac{3}{2}\right) = -2.691
   \]

2. \( \left(-\frac{3\sqrt{3}}{2}, \frac{3}{2}\right) \):
   \[
   f = \left(-\frac{3\sqrt{3}}{2}\right)^2 + \left(\frac{3}{2}\right)^2 - 3\left(-\frac{3\sqrt{3}}{2}\right) - \left(-\frac{3\sqrt{3}}{2}\right)\left(\frac{3}{2}\right) = 20.691
   \]

3. \( (0, -3) \):
   \[
   f = 0^2 + (-3)^2 - 3(0) - 0(-3) = 9.
   \]

4. \( (2,1) \):
   \[
   f = 2^2 + 1^2 - 3(2) - (2)(1) = 4 + 1 - 6 - 2 = -3.
   \]

Since we are taking absolute values:
\[
| -2.691 | = 2.691, \quad | 20.691 | = 20.691, \quad | -3 | = 3, \quad | 9 | = 9.
\]

Thus, the absolute maximum is:
\[
20.691
\]
and the absolute minimum is:
\[
2.691
\]

\end{parts}




\newpage
\question Consider the parabola
\[
y^2 = 4x,
\]
and let \( P = (2, 1) \) be a fixed point. The goal is to find the point on the parabola, \( Q \), that is closest to \( P \) using the method of Lagrange multipliers.

\begin{parts}
\part Define the squared distance function from a point \( (x, y) \) on the parabola to \( P \).

The squared distance function is:
\[
D(x, y) = \sqrt{(x - 2)^2 + (y - 1)^2}
\]

\part Use the method of Lagrange multipliers to determine the candidate point(s) on the parabola that minimizes the distance between \( P \) and the parabola.

The constraint is \( y^2 = 4x \). Define:
\[
g(x, y) = y^2 - 4x = 0.
\]
The gradients of \( D \) and \( g \) must satisfy:
\[
\nabla D = \lambda \nabla g.
\]
Compute the gradients:
\[
\nabla D = (2(x - 2), 2(y - 1)),
\]
\[
\nabla g = (-4, 2y).
\]
Set up the equations:
\[
2(x - 2) = \lambda (-4),
\]
\[
2(y - 1) = \lambda (2y).
\]
Solve for \( \lambda \):
From the first equation:
\[
\lambda = \frac{2(x - 2)}{-4} = -\frac{x - 2}{2}.
\]
From the second equation:
\[
\lambda = \frac{2(y - 1)}{2y} = \frac{y - 1}{y}.
\]
Set the two expressions for \( \lambda \) equal:
\[
-\frac{x - 2}{2} = \frac{y - 1}{y}.
\]
Cross-multiply and simplify:
\[
-(x - 2)y = 2(y - 1),
\]
\[
-xy + 2y = 2y - 2,
\]
\[
-xy = -2,
\]
\[
xy = 2.
\]
Use the constraint \( y^2 = 4x \) to substitute \( x = \frac{y^2}{4} \):
\[
\left(\frac{y^2}{4}\right)y = 2,
\]
\[
\frac{y^3}{4} = 2,
\]
\[
y^3 = 8,
\]
\[
y = 2.
\]
Substitute \( y = 2 \) into \( y^2 = 4x \):
\[
4 = 4x \implies x = 1.
\]
Thus, the candidate point is \( Q = (1, 2) \).

\part Show that the line joining \( P \) to \( Q \) is normal to the parabola.

The slope of the line joining \( P = (2, 1) \) and \( Q = (1, 2) \) is:
\[
m_{\text{line}} = \frac{2 - 1}{1 - 2} = -1.
\]
The slope of the tangent to the parabola \( y^2 = 4x \) at \( Q = (1, 2) \) is found by differentiating implicitly:
\[
2y \frac{dy}{dx} = 4 \implies \frac{dy}{dx} = \frac{2}{y}.
\]
At \( Q = (1, 2) \), the slope is:
\[
m_{\text{tangent}} = \frac{2}{2} = 1.
\]
Since \( m_{\text{line}} \cdot m_{\text{tangent}} = (-1)(1) = -1 \), the line is normal to the parabola.
\newpage
\part Sketch of the parabola with points Q, P, the tangent line to the parabola at Q, and the line connecting P and Q is shown in Figure 1:
\begin{figure}
    \centering
    \includegraphics[width=0.7\linewidth]{image_2025-02-17_222941545.png}
    \caption{Sketch}
    \label{fig:enter-label}
\end{figure}
\end{parts}
\end{questions}
\end{document}