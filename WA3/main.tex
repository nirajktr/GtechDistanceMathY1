\documentclass[11pt]{exam}
\usepackage{amsmath}
\usepackage{amssymb}
\usepackage{array}
\usepackage[bmargin=1in, tmargin=0.75in, lmargin=0.75in, rmargin=0.75in]{geometry}

\begin{document}
\thispagestyle{empty}


\begin{center}
    \Large{MATH 1554 QH, Written Assignment 3}

    \vspace{0.5cm}
    \large{Niraj Khatri}

    \vspace{0.5cm}
    \small{October 2024}
\end{center}

\vspace{1cm}

\begin{questions}

  % Question 1
  \question[5] Consider the matrix $A$ below, where $k$ is a real number.
    \[
    A = \begin{bmatrix} 
    -2 & 2 & 2 \\ 
    2 & k & -1 \\ 
    -6 & 3 & 5 
    \end{bmatrix}
    \]
    \vspace{0.2cm}
  \begin{parts}
    \part
   The determinant of matrix $A$ can be computed using cofactor expansion along the first row:
    \[
    \det(A) = (-2) \cdot \det\begin{pmatrix} k & -1 \\ 3 & 5 \end{pmatrix} - 2 \cdot \det\begin{pmatrix} 2 & -1 \\ -6 & 5 \end{pmatrix} + 2 \cdot \det\begin{pmatrix} 2 & k \\ -6 & 3 \end{pmatrix}
    \]
    Now, we compute each \(2 \times 2\) determinant:
    \[
    \det\begin{pmatrix} k & -1 \\ 3 & 5 \end{pmatrix} = k \cdot 5 - (-1) \cdot 3 = 5k + 3
    \]
    \[
    \det\begin{pmatrix} 2 & -1 \\ -6 & 5 \end{pmatrix} = 2 \cdot 5 - (-1) \cdot (-6) = 10 - 6 = 4
    \]
    \[
    \det\begin{pmatrix} 2 & k \\ -6 & 3 \end{pmatrix} = 2 \cdot 3 - k \cdot (-6) = 6 + 6k = 6k + 6
    \]
    Substitute these into the cofactor expansion:
    \[
    \det(A) = (-2) \cdot (5k + 3) - 2 \cdot 4 + 2 \cdot (6k + 6)
    \]
    Simplifying:
    \[
    \det(A) = -2(5k + 3) - 8 + 2(6k + 6)
    \]
    \[
    \det(A) = -10k - 6 - 8 + 12k + 12
    \]
    \[
    \det(A) = 2k - 2
    \]
    Thus, the determinant of \( A \) is:
    \[
    \det(A) = 2(k - 1)
    \]
    \part
    A matrix is singular if its determinant is zero. From part (a), we have:
    \[
    \det(A) = 2(k - 1)
    \]
    Setting this equal to zero:
    \begin{align*}
        2(k - 1) &= 0 \\
        k - 1 &= 0 \\
        k &= 1
    \end{align*}
    Thus, the matrix $A$ is singular when $k$ = 1.
    \part
When \( k = 1 \), the matrix \( A \) becomes:
\[
A = \begin{bmatrix}
-2 & 2 & 2 \\
2 & 1 & -1 \\
-6 & 3 & 5
\end{bmatrix}
\]
To find the eigenvalues, we need to solve the characteristic equation:
\[
\det(A - \lambda I) = 0
\]
Substituting \( A \) and the identity matrix \( I \), we have:
\[
\det\left( \begin{bmatrix}
-2 & 2 & 2 \\
2 & 1 & -1 \\
-6 & 3 & 5
\end{bmatrix} - \lambda \begin{bmatrix}
1 & 0 & 0 \\
0 & 1 & 0 \\
0 & 0 & 1
\end{bmatrix} \right) = 0
\]
This simplifies to:
\[
\det\begin{bmatrix}
-2 - \lambda & 2 & 2 \\
2 & 1 - \lambda & -1 \\
-6 & 3 & 5 - \lambda
\end{bmatrix} = 0
\]

Now, we can calculate the determinant. Using the cofactor expansion along the first row, we have:
\[
\begin{aligned}
\det\begin{bmatrix}
-2 - \lambda & 2 & 2 \\
2 & 1 - \lambda & -1 \\
-6 & 3 & 5 - \lambda
\end{bmatrix} &= (-2 - \lambda) \det\begin{bmatrix}
1 - \lambda & -1 \\
3 & 5 - \lambda
\end{bmatrix} - 2 \det\begin{bmatrix}
2 & -1 \\
-6 & 5 - \lambda
\end{bmatrix} + 2 \det\begin{bmatrix}
2 & 1 - \lambda \\
-6 & 3
\end{bmatrix} \\
&= (-2 - \lambda) \left( (1 - \lambda)(5 - \lambda) + 3 \right) - 2 \left( 2(5 - \lambda) + 6 \right) + 2 \left( 6 - 6(1 - \lambda) \right).
\end{aligned}
\]

Calculating the determinants of the 2x2 matrices:
1. 
\[
\det\begin{bmatrix}
1 - \lambda & -1 \\
3 & 5 - \lambda
\end{bmatrix} = (1 - \lambda)(5 - \lambda) + 3 = -\lambda^2 + 6\lambda + 2
\]
2. 
\[
\det\begin{bmatrix}
2 & -1 \\
-6 & 5 - \lambda
\end{bmatrix} = 2(5 - \lambda) + 6 = 16 - 2\lambda
\]
3. 
\[
\det\begin{bmatrix}
2 & 1 - \lambda \\
-6 & 3
\end{bmatrix} = 2 \cdot 3 + 6(1 - \lambda) = 6 + 6 - 6\lambda = 12 - 6\lambda
\]

Substituting these back, we expand and simplify:
\[
\begin{aligned}
&= (-2 - \lambda)(-\lambda^2 + 6\lambda + 2) - 2(16 - 2\lambda) + 2(12 - 6\lambda) \\
&= (2 + \lambda)(\lambda^2 - 6\lambda - 2) - 32 + 4\lambda + 24 - 12\lambda \\
&= (2 + \lambda)(\lambda^2 - 6\lambda - 2) - 8\lambda - 8.
\end{aligned}
\]

Finally, we can factor the characteristic polynomial obtained:
\[
\lambda (\lambda - 2)^2 = 0
\]

The solutions to this equation are:
\[
\lambda_1 = 0, \quad \lambda_2 = 2
\]
with \( \lambda_2 = 2 \) having an algebraic multiplicity of 2.


\part Construct the eigenbasis for each eigenvalue when \( A \) is singular.

When \( A \) is singular, one of the eigenvalues is \( \lambda_1 = 0 \). To construct the eigenbasis for this eigenvalue, we solve the equation:

\[
(A - 0I) \mathbf{v} = A \mathbf{v} = 0
\]
where \( \mathbf{v} \) is the eigenvector corresponding to \( \lambda_1 = 0 \).

Substituting the given matrix \( A \):

\[
A = \begin{bmatrix}
-2 & 2 & 2 \\
2 & 1 & -1 \\
-6 & 3 & 5
\end{bmatrix}
\]

We solve the system:

\[
\begin{bmatrix}
-2 & 2 & 2 \\
2 & 1 & -1 \\
-6 & 3 & 5
\end{bmatrix}
\begin{bmatrix}
v_1 \\
v_2 \\
v_3
\end{bmatrix}
= 
\begin{bmatrix}
0 \\
0 \\
0
\end{bmatrix}
\]

This gives the following system of equations:

\[
-2v_1 + 2v_2 + 2v_3 = 0
\]
\[
2v_1 + v_2 - v_3 = 0
\]
\[
-6v_1 + 3v_2 + 5v_3 = 0
\]

Solving the system:

From the first equation:
\[
-2v_1 + 2v_2 + 2v_3 = 0 \implies v_1 = v_2 + v_3
\]

Substitute \( v_1 = v_2 + v_3 \) into the second equation:
\[
2(v_2 + v_3) + v_2 - v_3 = 0 \implies 3v_2 + v_3 = 0 \implies v_3 = -3v_2
\]

Now substitute \( v_3 = -3v_2 \) into \( v_1 = v_2 + v_3 \):
\[
v_1 = v_2 + (-3v_2) = -2v_2
\]

Thus, the eigenvector corresponding to \( \lambda_1 = 0 \) is:
\[
\mathbf{v} = \begin{bmatrix} v_1 \\ v_2 \\ v_3 \end{bmatrix} = v_2 \begin{bmatrix} -2 \\ 1 \\ -3 \end{bmatrix}
\]

Any scalar multiple of \( \begin{bmatrix} -2 \\ 1 \\ -3 \end{bmatrix} \) is an eigenvector for \( \lambda_1 = 0 \).

For the eigenvalue \( \lambda_2 = 2 \), we solve the equation:

\[
(A - 2I) \mathbf{v} = 0
\]

Substituting \( A - 2I \):

\[
A = \begin{bmatrix}
-2 & 2 & 2 \\
2 & 1 & -1 \\
-6 & 3 & 5
\end{bmatrix}, \quad 2I = 2 \begin{bmatrix}
1 & 0 & 0 \\
0 & 1 & 0 \\
0 & 0 & 1
\end{bmatrix}
\]

\[
A - 2I = \begin{bmatrix}
-2 & 2 & 2 \\
2 & 1 & -1 \\
-6 & 3 & 5
\end{bmatrix} - \begin{bmatrix}
2 & 0 & 0 \\
0 & 2 & 0 \\
0 & 0 & 2
\end{bmatrix} = \begin{bmatrix}
-4 & 2 & 2 \\
2 & -1 & -1 \\
-6 & 3 & 3
\end{bmatrix}
\]

Now, we solve the system:

\[
\begin{bmatrix}
-4 & 2 & 2 \\
2 & -1 & -1 \\
-6 & 3 & 3
\end{bmatrix}
\begin{bmatrix}
v_1 \\
v_2 \\
v_3
\end{bmatrix}
= 
\begin{bmatrix}
0 \\
0 \\
0
\end{bmatrix}
\]

This system gives the following equations:

\[
-4v_1 + 2v_2 + 2v_3 = 0
\]
\[
2v_1 - v_2 - v_3 = 0
\]
\[
-6v_1 + 3v_2 + 3v_3 = 0
\]

Solving the system:

From the second equation:

\[
2v_1 - v_2 - v_3 = 0 \implies v_3 = 2v_1 - v_2
\]

Substitute \( v_3 = 2v_1 - v_2 \) into the first equation:

\[
-4v_1 + 2v_2 + 2(2v_1 - v_2) = 0 \implies -4v_1 + 2v_2 + 4v_1 - 2v_2 = 0 \implies 0 = 0
\]

This equation simplifies to \( 0 = 0 \), we proceed to the third equation:

Substitute \( v_3 = 2v_1 - v_2 \) into the third equation:

\[
-6v_1 + 3v_2 + 3(2v_1 - v_2) = 0 \implies -6v_1 + 3v_2 + 6v_1 - 3v_2 = 0 \implies 0 = 0
\]

This equation also simplifies to \( 0 = 0 \).

The solution has free variables. Let \( v_1 = 1 \), and \( v_2 = 0 \). Then, \( v_3 = 2 \), which gives the eigenvector:

\[
\mathbf{v_1} = \begin{bmatrix} 1 \\ 0 \\ 2 \end{bmatrix}
\]

Similarly, if \( v_1 = 1 \), and \( v_2 = 2 \), then \( v_3 = 0 \), which gives the second eigenvector:

\[
\mathbf{v_2} = \begin{bmatrix} 1 \\ 2 \\ 0 \end{bmatrix}
\]

The set of independent vectors corresponding to $\lambda_2$ = 2 is given by:

\[
\left\{ \begin{bmatrix} 1 \\ 0 \\ 2 \end{bmatrix}, \begin{bmatrix} 1 \\ 2 \\ 0 \end{bmatrix} \right\}
\]


\end{parts}

  \newpage

  % Question 2
  \question[5] Consider the Markov chain given.
  \vspace{0.2cm}
  \begin{parts}
    \part
    The transition matrix $P$ can be expressed as:
    \[
    P = \begin{bmatrix}
    0.5 & 0.1 & 0 \\
    0.5 & 0.5 & 1 \\
    0 & 0.4 & 0
    \end{bmatrix}
    \]
    
    \part
    A Markov chain is regular if, for some power \( P^k \), all entries are positive. To verify this, let's compute \( P^2 \):
    \[
    P^2 = P \times P = \begin{bmatrix}
    0.5 & 0.1 & 0 \\
    0.5 & 0.5 & 1 \\
    0 & 0.4 & 0
    \end{bmatrix}
    \times
    \begin{bmatrix}
    0.5 & 0.1 & 0 \\
    0.5 & 0.5 & 1 \\
    0 & 0.4 & 0
    \end{bmatrix}
    = \begin{bmatrix}
    0.3 & 0.1 & 0.1 \\
    0.5 & 0.7 & 0.5 \\
    0.2 & 0.2 & 0.4
    \end{bmatrix}
    \]
    
    Since \( P^2 \) has all positive entries, the Markov chain is regular.

    \part
To find the steady-state vector \( \mathbf{q} \), we need to solve the equation:
\[
(P - I) \mathbf{x} = 0
\]
where \( I \) is the identity matrix. The matrix \( P - I \) is given by:
\[
P - I = \begin{bmatrix}
0.5 & 0.1 & 0 \\
0.5 & 0.5 & 1 \\
0 & 0.4 & 0
\end{bmatrix} - \begin{bmatrix}
1 & 0 & 0 \\
0 & 1 & 0 \\
0 & 0 & 1
\end{bmatrix} = \begin{bmatrix}
-0.5 & 0.1 & 0 \\
0.5 & -0.5 & 1 \\
0 & 0.4 & -1
\end{bmatrix}
\]

Now we set up the augmented matrix for the system:
\[
\begin{bmatrix}
-0.5 & 0.1 & 0 & | & 0 \\
0.5 & -0.5 & 1 & | & 0 \\
0 & 0.4 & -1 & | & 0
\end{bmatrix}
\]

Next, we row-reduce this matrix. The reduced row echelon form is:
\[
\begin{bmatrix}
1 & 0 & -0.5 & | & 0 \\
0 & 1 & -2.5 & | & 0 \\
0 & 0 & 0 & | & 0
\end{bmatrix}
\]

From the matrix , we can express the variables in terms of \( x_3 \):
\[
x_1 = 0.5x_3, \quad x_2 = 2.5x_3
\]

Letting \( x_3 = t \), we can write the general solution as:
\[
\mathbf{x} = t \begin{bmatrix}
0.5 \\
2.5 \\
1
\end{bmatrix}
\]

The basis for the solution space is given by the vector:
\[
\begin{bmatrix}
0.5 \\
2.5 \\
1
\end{bmatrix}
\]

To find the steady-state vector \( \mathbf{q} \), divide the vector by the sum of its entries:
\[
0.5 + 2.5 + 1 = 4
\]

Thus, the steady-state vector \( \mathbf{q} \) is:
\[
\mathbf{q} = \frac{1}{4} \begin{bmatrix}
0.5 \\
2.5 \\
1
\end{bmatrix} = \begin{bmatrix}
0.125 \\
0.625 \\
0.25
\end{bmatrix}
\]

\end{parts}



  \newpage

\end{questions}

\end{document}
